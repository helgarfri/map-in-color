\documentclass{article}

% Pakkar
\usepackage[utf8]{inputenc} % UTF-8 kóðun
\usepackage[T1]{fontenc}    % Leturkóðun fyrir íslenska stafi
\usepackage{graphicx}       % Fyrir myndir
\usepackage{listings}       % Fyrir kóðalista
\usepackage{xcolor}         % Fyrir sérsniðna liti
\usepackage{minted}         % Fyrir syntax highlighting

% Skilgreina sérsniðna liti fyrir syntax highlighting
\definecolor{keywordcolor}{rgb}{0,0,0.8}
\definecolor{commentcolor}{rgb}{0,0.5,0}
\definecolor{stringcolor}{rgb}{0.58,0,0.82}

% Breyta myndamerki yfir á íslensku
\renewcommand{\figurename}{Mynd}

% --- Removed custom date logic and replaced with empty date ---
\title{MIC Documentation}
\author{Helgi Freyr Davíðsson}
\date{} % No date

% Kóðalista stillingar
\lstset{
    language=[x86masm]Assembler, 
    basicstyle=\footnotesize\ttfamily, 
    numbers=left, 
    numberstyle=\tiny, 
    stepnumber=1, 
    numbersep=5pt, 
    backgroundcolor=\color{white}, 
    showspaces=false, 
    showstringspaces=false, 
    showtabs=false, 
    tabsize=2, 
    captionpos=b, 
    breaklines=true, 
    breakatwhitespace=false, 
    keywordstyle=\color{keywordcolor}\bfseries,
    commentstyle=\color{commentcolor}\itshape, 
    stringstyle=\color{stringcolor}, 
    frame=single, 
    framerule=0.5pt,
    framesep=2mm,
    escapeinside={\%*}{*)},
    morekeywords={MOV, ADD, SUB, MUL, DIV, PUSH, POP, CALL, RET, JMP, JE, JNE, JG, JL, JGE, JLE},
}

\begin{document}

\maketitle

\section{Introduction}

This document is a collection of all the documentation for the MIC (Map in Color) project. The project is a web application that allows users to create and share custom maps displaying chloropleth data from a CSV file. Technologies used in the project include React for the front end, Node.js for the back end, and Postgres for the database.

\section{Creating a Map}
To create a map in MIC, the user must navigate to the data integration page. They can select “create a new map” from the header or the sidebar menu. The user will get a pop-up selection where they can select which type of map they would like to use (e.g. World Map, US Map, Europe Map, etc.). Then the user will be taken to the data integration page where they can upload a CSV file with the data to be displayed on the map. 

They have the option of downloading a template with all the countries or regions their selected map contains. The CSV file must contain a column with the name of the country or region and a column with the data to be displayed. If the CSV file is not correctly formatted (e.g. missing columns, wrong region names, or incorrect data types), the user will receive an error log with the line number and details of the issues. 

If all the data is valid, the user can proceed to define custom ranges for the data. The user also has an option to generate suggested ranges with the desired number of ranges. They can name each range and select a color scheme. Once all ranges are valid (no overlapping ranges), the user can generate groups that will appear in the map preview below. The user can also customize the map theme (either by selecting from a preset of themes or by choosing custom font color, ocean color, and unassigned countries color). 

After that, the user will get a map settings pop-up where the title of the map is defined (with an option to hide the title), a map description, map tags, and visibility options (public or private). Once the map is created, the user can navigate to the map view page to see all the information about the map and share it with others.

\end{document}
